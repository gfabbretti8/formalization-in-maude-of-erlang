\documentclass{article}[12pt,a4paper]

%!TEX root = main.tex

\def\finex{{\unskip\nobreak\hfil
\penalty50\hskip1em\null\nobreak\hfil{\Large $\diamond$}
\parfillskip=0pt\finalhyphendemerits=0\endgraf}}

\newenvironment{lmref}[1]{%
	\vspace*{0.3cm}
	\noindent {\bf  Lemma #1.}}

\newenvironment{thref}[1]{%
	\vspace*{0.3cm}
	\noindent {\bf  Theorem #1.}}

\newenvironment{corref}[1]{%
	\vspace*{0.3cm}
	\noindent {\bf  Corollary #1.}}


\newcommand{\picalc}{$\pi$-calculus~}
\newcommand{\ms}[1]{\mathsf{#1}}
\newcommand{\ctx}{\mathtt{C}}
\newcommand{\co}[1]{\overline{#1}}
\newcommand{\proj}[1]{\mathtt{proj}(#1)}
\newcommand{\projs}[1]{\mathtt{proj_n}(#1)}
\newcommand{\lts}[1]{\xrightarrow{#1}}
\newcommand{\comm}[1]{\textcolor{blue}{#1}}
\newcommand{\red}[1]{\textcolor{red}{#1}}
\newcommand{\key}{\mathtt{keys}}
\newcommand{\flts}[1]{\overset{#1}\twoheadrightarrow}
\newcommand{\rev}{^{\bullet}}
%\newcommand{\forw}{_{\twoheadrightarrow}}
\newcommand{\eq}{\sim} 
\newcommand{\env}{\theta} 
\newcommand{\ex}{e} 
\newcommand{\proc}[3]{\langle #1,#2,#3\rangle } 
\newcommand{\erl}[1]{\hookrightarrow_{#1}} 
\newcommand{\mem}[2]{[#1\;;#2]}
\newcommand{\bl}[1]{\textcolor{blue}{#1}}
\newcommand{\rd}[1]{\textcolor{red}{#1}}
\newcommand{\mtt}[1]{\mathtt{#1}}
\newcommand{\pid}{\mathtt{pid}}
\newcommand{\procl}[4]{\langle #1,#2,#3,#4\rangle } 
\newcommand{\con}{\equiv}
\newcommand{\conk}{\equiv_k}
\newcommand{\extcon}{\equiv_c}
\newcommand{\de}{\delta}
\newcommand{\G}{\Gamma}
\newcommand{\logg}[1]{\rightharpoonup_{#1}}
\newcommand{\rlogg}[1]{\leftharpoondown_{#1}}
\newcommand{\la}{\lambda}
\newcommand{\col}[1]{\mathtt{col}(#1)}
\newcommand{\rel}{\mathcal{R}}
\newcommand{\conc}{\smile_c}
\newcommand{\nil}{{\bf{0}}}
%\newcommand{\res}{\nu}
\newcommand{\res}[1]{\nu #1\,}
\newcommand{\out}[1]{\langle #1\rangle}
\newcommand{\cont}{\triangleright}
\newcommand{\sub}[2]{\{#1/#2\}}
\newcommand{\op}{op_n}
\newcommand{\pidd}{Pid}
\newcommand{\fmod}{\rightarrowtail}
\newcommand{\enc}[1]{\llparenthesis #1 \rrparenthesis}
\newcommand{\blt}{\bullet}
\newcommand{\str}{C}
\newcommand{\term}[1]{T[#1]}
\newcommand{\en}[1]{\Lbag #1 \Rbag}
\newcommand{\signal}[1]{\{ #1\}}
\newcommand{\systset}{\mathbb{S}}
\newcommand{\confset}{\mathbb{C}}
\newcommand{\set}[1]{\{ #1\}}
\newcommand{\node}[3]{#1,#2\mkern-3mu:\mkern-3mu[\mkern-6mu[#3]\mkern-6.2mu]}


\usepackage{listings}
\usepackage[version=3]{mhchem} % Package for chemical equation typesetting
\usepackage{siunitx} % Provides the \SI{}{} and \si{} command for typesetting SI
% units
\usepackage{fancyvrb}
\usepackage[hidelinks]{hyperref}
\usepackage{breakurl}             % Not needed if you use pdflatex only.
\usepackage{underscore}           % Only needed if you use pdflatex.
\usepackage{microtype}%if unwanted, comment out or use option "draft"
\usepackage{amssymb}
\setcounter{tocdepth}{3}
\usepackage{graphicx}
\usepackage{listings}
\usepackage{color}
\usepackage{rotating}
\usepackage{todonotes}
\usepackage{mathpartir}
\usepackage{url}
\usepackage{tikz}
\usepackage{amsmath}
\usepackage{stmaryrd}
\usepackage{amsthm}
\usepackage{float}
\usepackage{hyperref}
\usepackage{thm-restate}


\renewcommand{\labelenumi}{\alph{enumi}.} % Make numbering in the enumerate environment by letter rather than number (e.g. section 6)

\theoremstyle{definition}
\newtheorem{example}{Example}[section]
\newtheorem{theorem}{Theorem}
\newtheorem{lemma}{Lemma}
\newcommand{\paral}{\;|\;}
\newcommand{\cons}{\mbox{:}}

\usetikzlibrary{calc,decorations.pathmorphing,shapes}
\newcounter{sarrow}
\newcommand\blts[1]{%
  \stepcounter{sarrow}%
  \mathrel{\begin{tikzpicture}[baseline= {( $ (current bounding box.south) + (0,-0.5ex) $ )}]
      \node[inner sep=.5ex] (\thesarrow) {$\scriptstyle #1$};
      \path[draw,<-,decorate,
      decoration={zigzag,amplitude=1.2pt,segment length=1.5mm,pre=lineto,pre length=6pt}]
      (\thesarrow.south east) -- (\thesarrow.south west);
    \end{tikzpicture}}%
}

\begin{document}

\title{Generation of a Reversible Semantics for Erlang in Maude (Technical Report)}

\author{Giovanni Fabbretti, Ivan Lanese}
\date{}

\maketitle % typeset the header of the contribution 

\abstract{
  In recent years, reversibility in concurrent settings has attracted quite
  some interest thanks to its many applications in different areas, like error-recovery,
  debugging, biological systems. Some of the formalisms where it has been
  investigated are: petri-nets, process algebras and real programming languages.
  Nonetheless, all the
  attempts made so far suffer from the same limitation:
  they have been studied ad-hoc. To address this limit Lanese et al.~have recently
  proposed a new general method to derive a reversible semantics starting from a
non-reversible one. The general method though lacks an implementation
that proves its feasibility in practice. The
aim of this work is to provide such an implementation and
to apply it to a case study on the Erlang programming language. }

\section{Introduction}


Reversibility is the capability to execute a program both in a forward
and backward manner. While reversibility is well-understood for
sequential systems the same cannot be said for concurrent
ones. Indeed, in sequential systems, actions can be undone in reverse
order of completion.  The major difficulty while reversing the
execution of a concurrent system is that the order of actions is no
more a total order, since execution of concurrent actions may overlap.
To tackle this problem Danos and Krivine in~\cite{DanosK04} propose a
notion of \emph{causal-consistent reversibility}, which aims at
establishing a way to undo actions in concurrent systems. Causal
consistency states that an action can be undone iff all of its
consequences, if any, have been undone beforehand. Notably, this
definition does not need a (temporal) total order of actions, but
builds on a notion of causality.

Following their seminal work, reversibility has been studied in other formalisms
as well, among which we find: CCS~\cite{DanosK04,PhillipsU07}, $\pi$-calculus~\cite{CristescuKV13}, higher-order $\pi$-calculus~\cite{LaneseMS16},
$\mu$Klaim~\cite{GiachinoLMT17}, Petri-nets~\cite{PhilippouP18,MelgrattiMU20}, etc.

Another stream of research, investigating reversibility as a debugging technique for concurrent systems,
has risen after \cite{GiachinoLM14}, where Giachino et al.~proposed to
equip a concurrent debugger with reversible primitives.
The approach by Giachino et al.~has subsequently been applied in the context of a real programming language, namely Erlang~\cite{LaneseNPV18,Lanese0PV18,Gonzalez-AbrilV21,FabbrettiLS21}.

However, as anticipated, all the works cited so far suffer from the same limitation: reversibility has always been
devised manually. Given a forward non-reversible semantics, the authors manually
derived an extension of its forward semantics, enabling reversibility, and the corresponding backward semantics, defining reverse execution.
Deriving a reversible semantics manually presents some limitations: the
process is error-prone, it does not scale well to other formalisms, and lacks
uniformity - i.e., the same properties must be proved for each formalism.

To address this problem, Lanese and Medic~proposed in~\cite{LaneseM20} a
general method to
automatically derive a causal-consistent reversible semantics starting from a non-reversible one. The
advantages of having an automatic method are symmetric to the disadvantages
listed above, i.e., the method: is not error-prone; scales well; is uniform. The key idea behind the general method is to capture
causal dependencies, needed for the definition of causal-consistent reversibility, in terms of resources consumed and produced. The non-reversible
semantics taken in input must be a reduction semantics and the entities on the
left-hand side of a rule are seen as the resources to be \emph{consumed} to
\emph{produce} the resources on the right-hand side - i.e., the new entities.
As an example, even without knowing the details that will be explained later on, let
us consider the following rule. 
\begin{equation}\label{samplerule}
\langle p_1, \theta, p_2~\ms{!}~hello, me\rangle \rightarrow \langle p_1, \theta,
hello, me\rangle \paral\langle p_1, p_2,hello \rangle
\end{equation}
On the left we have a process ready to send a
message, when the rule is fired the process is consumed to produce the message
$\langle p_1, p_2,hello \rangle$ and the evolution of the process itself after the send.

In order to define the reversible semantics, we need two ingredients: keys and memories.
Keys are attached to entities, so to distinguish entities with the same form but different history.
Memories are used to track past states, that would normally be lost during the computation, so that
they can be restored during backward execution.

In~\cite{LaneseM20} the approach is only described theoretically and
no implementation is provided. Actually, the method relies on
reversing instances of rules such as the one above, which are infinitely many in almost any
non trivial formalism, what makes not immediately clear that an
implementation could exist. Our work aims at showing that indeed the
method can be implemented and used in practice.  The key idea here is
that the instances of rules can be captured by a finite (and
frequently small) number of rule schemas, and the approach can be
applied to schemas. In order to do this, we use Maude~\cite{maude} to represent the reduction semantics (both the non-reversible one in input and the reversible one we generate) as well as for the tool generating the reversible semantics.
To further prove the practicality of the
approach, we apply it to a case study on the Erlang programming
language.

The main contributions of this work are:
\begin{itemize}
  \item a novel formalization of Erlang semantics using
    Maude;
  \item the implementation of a tool that automatically derives a reversible
    semantics starting from a non reversible one.
\end{itemize}

The rest of the report provides the reader with the required
background in Section~\ref{sec:background}. In
Section~\ref{sec:formalizing-erlang} we will present the formalization of
the Erlang semantics in Maude and in Section~\ref{sec:generating} we discuss how the reversible semantics
is generated. In Section~\ref{sec:ongoing-work} we discuss some
ongoing work. Finally, in Section~\ref{sec:conclusion} we give some
conclusions and we hint at possible future directions.

All the code discussed in this report is publicly available
at~\cite{erl-maude-repo}. 

\section{Background}\label{sec:background}

\subsection{Erlang: Syntax and Semantics}

Erlang is a functional and concurrent programming language. First
introduced in 1986 by Ericsson, it has gained quite a lot of
popularity since then.  Today it is widely used and mostly appreciated
because it is easy to learn, provides useful abstractions for
concurrent and distributed programming, and because of its support for
highly-available systems. Erlang implements the actor model~\cite{Hewitt73}, a
concurrency model based on message passing. In the actor model, each
process is an actor that can interact with other actors
only through the exchange of messages, no memory is shared. Indeed,
central in Erlang are the $\ms{send}$, $\ms{receive}$ and $\ms{spawn}$
operations, to, respectively, send a message, receive a message, and
create a new actor. Actors are identified by a pid (process identifier) and have a queue of messages which have arrived but have not yet been processed.
An actor evaluates an expression, and has an environment to store variable bindings.

The rest of this section provides the reader with a basic understanding
of the Erlang programming language. We begin by illustrating its syntax, depicted in
Fig.~\ref{ErlangSyntax}\footnote{We support the functional and concurrent fragment of the
  real Erlang language.}. 

An Erlang program can be seen as a collection of modules, where a module is a sequence of function definitions. Each function is uniquely identified by its name and by the
number of formal parameters. Each function may be specified by cases via multiple definitions. The correct definition for each invocation is selected by pattern matching on parameters. The body of each definition is represented
by a sequence of expressions. In the following we denote an expression with $e$
and sequences of expressions as $e_1,\ldots,e_n$  - sequences of other
syntactical elements are represented in the same manner.  

Ground values in Erlang are: atoms (which are identifiers that either
begin with a lowercase or are enclosed by quotes), integers and
strings, as well as compositions of values using tuple and list
constructs. We range over ground values using $v$. Tuples are denoted
as $\{v_1,\ldots,v_n\}$ and lists are denoted as
$[v_1|v_2]$, where $v_1$ is the head and $v_2$ the tail.


\begin{figure}[t]
  \begin{center}
    $
    \begin{array}{rcl@{~~~~~~}l}

      program & ::= & mod_1  \dots  mod_n \\
      mod & ::= & fun\_def_1  \dots fun\_def_n  \\
      fun\_def & ::= & Atom([patterns]) \to exprs.\\
      pat & ::= & b\_value \mid Var \mid ~'\{'[patterns]'\}' \mid ~
                  '['[patterns|patterns] ']'\\
      patterns & ::= & pat ~\{ ','patterns\} \\
      exprs & ::= & expr ~\{ ',' exprs\} \\
      expr & ::= & b\_value \mid Var \mid ~'\{'[exprs]'\}' \mid ~'[' [exprs|expr] ']' \\
                    & \mid & \ms{case}~expr~\ms{of}~clseq~\mathsf{end} \mid
                             \ms{receive}~clseq~\mathsf{end} \mid expr ~ ! ~ expr \\
                    & \mid & pat = expr \mid
                             [Mod\cons]expr([exprs]) \\
      b\_value & ::= & Atom \mid Char \mid Float \mid Integer \mid String \\
      clseq & ::= & pat  \to exprs ~ \{ ';' pat \to exprs  \} \\
    \end{array}
    $
  \end{center}
  \caption{Language syntax} \label{ErlangSyntax}
\end{figure}

Variables can store ground values. Variables, e.g., $X,Age$, start with a capital letter and
are not enclosed in quotes.
Patterns, denoted by $pat$, are like the ground values,
but also admit the presence of variables. Patterns are used for pattern matching in the following contexts: (i) in the matching operation $e_1 = e_2$, (ii) in case statements $\ms{case}~e~\ms{of}~clseq~\ms{end}$ to choose the branch
to evaluate according to the shape of the incoming data,
(iii) in receive
statements $\ms{receive}~pat_1 \rightarrow exprs_1; \ldots; pat_n
\rightarrow exprs_n~\ms{end}$, to analyze the shape of the received message, and
(iv) in function
definitions, to define the formal parameters. 

We start by explaining the match operation. First, the expression $e_2$ on
the right-hand side is evaluated until it becomes a ground value,
occurrences of free variables, if any, raise an exception.
Then, the expression on the left-hand side, $e_1$, is evaluated until
it becomes a pattern, or a ground value in case no free variables
occur in it. Then the
two elements are matched against each other. Each free variable of the
left-hand side is bound to the corresponding ground value of the right-hand
side, ground values in corresponding position should coincide: if a mismatch occurs then an exception
is raised. If no mismatch occurs then the operation evaluates to the ground
value of the right-hand side and the environment is updated with the new bindings.

In the $\ms{case}$ construct, the expression $e$ must evaluate to a ground value, then it is matched against the patterns, from
top to bottom, until one that matches is found. When a match is found the environment
is enriched with the new bindings and the corresponding sequence of expressions evaluated, if
no match is found an exception is raised.

The behaviour of the $\ms{receive}$ is similar to the one of the
$\ms{case}$, with the only difference that messages in the queue of
the process are tried as ground values till the match succeeds.
When a match is found the corresponding branch
is selected.  Contrary to the $\ms{case}$, if no match is found
then the process suspends.

Despite being - mostly - functional Erlang
admits some imperative operations that produce side-effects, like the receive
above, spawning a new process, and sending a message. 

The syntax of message send is $e_1!e_2$, where $e_1$ must
evaluate to the pid of the receiver process and $e_2$ must evaluate to the ground value that represents the payload of the message. The expression itself evaluates to
the payload and, as a side-effect, the message is sent.

The $\ms{spawn}$ primitive creates a new process; it takes as argument
the function $f$ that the new process will execute, together with the
parameters for $f$ - if any. The spawn returns the
pid of the newly created process and, as a side effect, a new process is created.

Finally, the function $\ms{self}$ returns the process id of the process
who invoked it.

Self, send, spawn and receive are the concurrent features, offered by Erlang, that we support in this work.

\subsection{Maude}

Maude~\cite{maude} is a programming language that efficiently implements rewriting logic~\cite{MeseguerMS96}.
Rewriting logic can be seen as a framework that unifies equational
logic with semantics rules.

Formally, a rewriting logic is a tuple $(\Sigma, E, R)$, where $\Sigma$
represents a collection of typed operators, $E$ a set of equations among the operators, and $R$ is the set of
semantics rules.

Using a rewriting logic is quite convenient to formalize the
semantics of a language as it provides the benefits of using both an equational theory and rewriting rules.

On the one hand, the equational side of rewriting logic is well-suited to define the deterministic part of the model, where
we define equivalence classes over terms. More precisely we say that two terms
$v$ and $u$ are equivalent if under a set of equations $E$ we can prove $E \vdash
v = u$. Equations can be conditional and conditions can be either the
membership of the term to some kind or other equations.

% Rules
On the other hand, the rewriting rules are well-suited to define the concurrent
(non-deterministic) part of the programming language. The set of rules $R$
specifies how to rewrite a (parametrized) term $t$ to another term $t'$.
Rewriting rules, like equations, can be conditional and as condition they can
use membership, equations, as well as other rewriting rules.

In other words the equational theory defines which terms define the same states
of a system, only using different syntactical elements, while the rewriting rules
define how the system can evolve and transit from one state to another.

Let us now consider the module in Fig.~\ref{fig:bool}, that is an example of a Maude module that implements Booleans together with their classic operations.

\begin{figure}[t]
\begin{verbatim}
fmod BOOL is
  sort Bool .
  op true : -> Bool [ctor] .
  op false : -> Bool [ctor] .

  op _and_ : Bool Bool -> Bool [assoc comm prec 55] .
  op _or_ : Bool Bool -> Bool [assoc comm prec 59] .
  op not_ : Bool -> Bool [prec 53] .

  vars A B C : Bool .

  eq true and A = A .
  eq false and A = false .
  eq A and A = A .
  eq not false = true .
  eq not true = false .
  eq A or B = not A and not B .
  eq not not A = A .
endfm
\end{verbatim}
  \label{fig:bool}
\end{figure}

First, the sort \verb+Bool+ is declared. Then, the values \verb+true+ and \verb+false+ are declared
as two constant operators of sort \verb+Bool+. Successively, the classic operations are
defined as functions that take in input some \verb+Bool+s and produce a \verb+Bool+ as a
result. For example, \verb+_and_ : Bool Bool -> Bool+ defines the \verb+and+
operator that takes in input two \verb+Bool+s and produces a \verb+Bool+. Finally, the semantics of these operators is given
by the equational theory defined at the bottom of the module. Equations are used
from left to right to normalize terms. For instance, the first equation,
\verb+eq true and A = A.+ is used to evaluate the \verb+and+ operator when the first
argument has been normalized to \verb+true+. For simplicity, this example does not include rewriting rules, memberships nor conditional equations.

As an additional example we show a rewriting rule generating the reduction
(\ref{samplerule}), where we omit the concrete instance of the
module environment for space reasons.

\begin{Verbatim}
< 1 | exp: 2 ! 'hello', env: {}, me: _ > => 
< 1 | exp : 'hello', env: {}, me: _ > ||
< sender: 1, receiver: 2, payload: 'hello' >
\end{Verbatim}

\begin{figure}[t]
\begin{verbatim}
 mod H is
    IL
    sorts SS .
    SSDS
    OPDS
    MAS
    EQS
    RLS
 endm .
\end{verbatim}
  
  \caption{A generic maude module.}
  \label{fig:maude-module}
\end{figure}

In general, a module in Maude has the shape depicted in Fig.~\ref{fig:maude-module}, where:
\verb+H+ is the module name; \verb+IL+ is the import list; \verb+SS+ is the set
of sort declarations; \verb+SSDS+ is the set of sub-sort declarations; \verb+OPDS+
is the set of operator declarations; \verb+MAS+ is the set of membership
declarations; \verb+EQS+ is the set of equations; \verb+RLS+ is the set of
rewriting rules. 

The transformation of the non-reversible semantics is defined in terms of a
program that takes in input the modules of the non-reversible semantics and
produces new modules, which define the reversible semantics.


\subsection{A General Approach to Derive a Reversible Semantics}\label{sec:gener-appr-derive-rev-sem}

%% One of the main limitations of the approach described in
%% Section~\ref{sec:fst-rev-sem} is that the causal dependencies introduced by each
%% of the supported primitives are derived ad-hoc. Obviously, this process is time
%% consuming, error-prone and does not scale very well, neither to new primitives
%% nor to other languages.

The following of this section summarizes the main ideas of~\cite{LaneseM20}
where Lanese et al.\ propose a methodology to automatically
derive a causal-consistent reversible semantics starting from a non-reversible one, provided that
the latter is modelled as a reduction semantics which satisfies some syntactic conditions.
%% One of the main advantages of using such method is that the format is general
%% enough that can be applied to any formalism that fits the criteria imposed.
%% Also, we are guaranteed that all the reversible semantics derived with this
%% method will all enjoy the same properties, which means that they must be proved only once. 

\subsubsection{Format of the Input Reduction Semantics}

We now describe the shape that the reduction semantics taken as input
must have.

First, the syntax must be divided in two levels: a lower level of entities on which there are no restrictions, and an upper level of systems of the following
form:
\[
  S::=P\;\paral \;\op(S_1,\ldots,S_n)\; \paral \nil
\]

where $\nil$ is the empty system, $P$ an entity of the lower level and $op(S_1,\ldots,S_n)$ an operator
composing entities. An entity of the lower-level could be, for example, a process of the system or
a message traveling the network.

% Talk about the schema of the forward rules, why it is necessary and how it is
% used to automatically derive the rev sem.

Second, the rules defining the operational semantics must fit the schema in Fig.~\ref{fig:forwardrules}.
The schema contains rules to: i) allow entities to interact with each other (\textsc{Scm-Act}); ii)
exploit structural congruence (\textsc{Eqv}); iii) allow single entities to execute inside a context (\textsc{Scm-Opn});
iv) execute two systems in parallel (\textsc{Par}). Notably, while (\textsc{Eqv}) and (\textsc{Par}) are rules that must belong to the semantics, (\textsc{Scm-Act}) and (\textsc{Scm-Opn}) are schemas, and the semantics may contain any number of instances of the schemas. Actually, rule (\textsc{Par}) is an instance of schema (\textsc{Scm-Opn}), highlighting that such an instance is required. As another example, reduction~\ref{samplerule} from the Introduction is an instance of schema (\textsc{Scm-Act}).
Also, notice that a notion of structural congruence on systems is assumed. 

\begin{figure}[t]
  {\footnotesize
    \begin{mathpar}
      \inferrule*[left=\footnotesize{(\textsc{Scm-Act})}]
      {\;}
      {P_1\paral \dots \paral P_n\fmod T[Q_1,\ldots, Q_m]}
      \and
      \inferrule*[left=\footnotesize{(\textsc{Eqv})}]
      {S\con_c S' \quad S\fmod  S_1 \quad S_1\con_c S'_1}
      {S'\fmod  S'_1}
      \and
      \inferrule*[left=\footnotesize{(\textsc{Scm-Opn})}]
      {S_i\fmod  S'_{i}}
      {\op(S_0,\ldots,S_i,\ldots,S_n)\fmod \op(S_0,\ldots,S'_{i},\ldots,S_n)}
      \and
      \inferrule*[left=\footnotesize{(\textsc{Par})}]
      { S\fmod  S' }
      {S\paral S_1\fmod  S'\paral S_1}
    \end{mathpar}}
  \caption{Required structure of the semantics in input; \textsc{Scm-} rules are schemas}
  \label{fig:forwardrules}
\end{figure}

\subsubsection{Methodology}\label{sec:methodology}

To obtain a forward reversible semantics, we need to track enough history and causality information to allow one to define a backward semantics exploiting it. First, the syntax of the
reduction semantics is updated as follows:\\
\[
\begin{array}{l}
  R   \; ::=\;  k:P \paral \op(R_1,\ldots,R_n)\paral \nil \paral \mem{R}{\str} \\[1ex]
  \str \; ::= \; T[k_1:\blt_1,\dots,k_m:\blt_m]
\end{array}
\]

Two modifications have been done. First, each entity of the system is tagged with
a key. Keys are used to distinguish identical processes with a different
history. Second, the syntax is updated with another production: memories. Memories have
the shape $\mu=[R;C]$, where $R$ is the configuration of the
system that gave rise to a forward step and $C$ is a context describing the structure of the system resulting from the forward step.
$C$ acts as a link between $R$ and the actual final configuration. In
other words, memories link different states of the entities. Moreover, they keep
track of past states of the system so that they can be restored.

Then, the rules of the non-reversible operational semantics are updated as depicted in
Fig.~\ref{fig:revforward}. Now each time a forward step is performed each resulting entity is tagged with a fresh
key, and a memory, connecting the old configuration with the new one, is produced.
E.g., the forward rule corresponding to reduction~(\ref{samplerule}):

\begin{Verbatim}
< 1 | exp: 2 ! 'hello', env: {}, me: _ > * key 0 => 
< 1 | exp : 'hello', env: {}, me: _ > * key 0 0 ||
< sender: 1, receiver: 2, payload: 'hello' > * key 1 0 ||
[ < 1 | exp: 2 ! 'hello', env: {}, me: _ > * key 0 ; @: key 0 0 || @: key: 1 0 ]
\end{Verbatim}

We can manipulate the rules without actually knowing them
because they fit the schema and they cannot have completely arbitrary forms. 

\begin{figure}[t]
  {\footnotesize
    \begin{mathpar}
      \inferrule*[left=\scriptsize{(\textsc{F-Scm-Act})}]
      % {P_1\para\!\! \dots\!\! \para P_n\fmod T[Q_1, \ldots, Q_m]  \qquad
      {j_1, \ldots ,j_m\text{ are fresh keys}}
      {k_1: P_1\paral \!\!\dots \!\!\paral k_n: P_n\flts{}T[j_1:Q_1,\ldots , j_m:Q_m]\paral \mem{k_1: P_1\paral \!\!\dots\!\! \paral k_n: P_n}{T[j_1:\blt_1,\ldots,j_m:\blt_m]}}
      \and
      \inferrule*[left=\scriptsize{(\textsc{F-Scm-Opn})}]
      {R_i\flts{}  R'_{i}\quad (\key(R_i')\setminus \key(R_i))\cap (\key(R_0,\ldots,R_{i-1},R_{i+1},\ldots,R_n)=\emptyset}
      {\op(R_0,\ldots,R_i,\ldots,R_n)\flts{} \op(R_0,\ldots,R'_{i},\ldots,R_n)}
      \and
      \inferrule*[left=\scriptsize{(\textsc{F-Eqv})}]
      {R\extcon R'  \quad R\flts{}  R_1 \quad R_1 \extcon R'_1 }
      {R'\flts{}  R'_1}
    \end{mathpar}}
  \caption{Forward rules of the uncontrolled reversible semantics}
  \label{fig:revforward}
\end{figure}

\begin{figure}[t]
  {\footnotesize
    \begin{mathpar}
      \inferrule*[left=\scriptsize{(\textsc{B-Scm-Act})}]
      {\mu=\mem{k_1: P_1\paral \dots \paral k_n: P_n}{T[j_1:\blt_1,\ldots,j_m:\blt_m]}}
      {T[j_1:Q_1,\ldots , j_m:Q_m]\paral \mu \blts{\quad}k_1: P_1\paral \dots \paral k_n: P_n}
      \and
      \mbox{}\hspace{-.5cm}
      \inferrule*[left=\scriptsize{(\textsc{B-Scm-Opn})}]
      {R'_{i}\blts{\quad}  R_i}
      { \op(R_0,\ldots,R'_{i},\ldots,R_n)\blts{\quad}\op(R_0,\ldots,R_i,\ldots,R_n)}
      \and
      \inferrule*[left=\scriptsize{(\textsc{B-Eqv})}]
      {R\extcon R' \ \ R\blts{\quad}  R_1 \ \  R_1\extcon R'_1}
      {R'\blts{\quad}  R'_1}
      % \inferrule*[left=\scriptsize{(\textsc{B-Eqv})}]
      % {\proj{R}\con \proj{R'} \ \ R\blts{\quad}  R_1 \ \  \proj{R_1}\con \proj{R'_1}}
      % {R'\blts{\quad}  R'_1}
    \end{mathpar}}
  \caption{Backward rules of the uncontrolled reversible semantics}
  \label{fig:revbackward}
\end{figure}

The backward rules, depicted in Fig.~\ref{fig:revbackward}, are symmetric to the
forward ones: when a memory $\mu$ and
the entities tagged with the keys in $C$ are both available then a backward step can be
performed and the old configuration $R$ can be restored.
E.g., the backward rule undoing the reduction~(\ref{samplerule}):

\begin{Verbatim}
< 1 | exp : 'hello', env: {}, me: _ > * key 0 0 ||
< sender: 1, receiver: 2, payload: 'hello' > * key 1 0 ||
[ < 1 | exp: 2 ! 'hello', env: {}, me: _ > * key 0 ; @: key 0 0 || @: key: 1 0 ]
=> 
< 1 | exp: 2 ! 'hello', env: {}, me: _ > * key 0

\end{Verbatim}

The reversible semantics produced by this approach captures causal dependencies
in terms of resources produced and consumed, since, thanks to the memory, a
causal link is created each time some entities are rewritten. We refer
to~\cite{LaneseM20} for the formal proof of the causal-consistency and of other relevant properties of
the reversible semantics. We also remark
that the semantics produced is uncontrolled~\cite{LaneseMS12}, i.e., if multiple (forward and/or backward) steps are enabled at the same time there is no policy on which one to choose.

\section{Formalizing Erlang in Maude}\label{sec:formalizing-erlang}

The work presented here has been strongly inspired
by~\cite{NeuhauberN07}, where the authors formalized the semantics of
Core Erlang~\cite{Car01} to do model-checking on it. While our final semantics is quite
different from the one they presented (the most notable differences are that we formalize
full Erlang and the use of labels) we were still able to re-use some of their
modules and some of their ideas, like the internal representation of
ground values, which greatly simplified the formalization task.

Our formalization of the semantics follows the style of the one
in~\cite{Gonzalez-AbrilV21}, with some differences which we will soon discuss. As
in~\cite{Gonzalez-AbrilV21}, the semantics has two
layers: one for expressions and one for the system. This division is quite convenient for the formalization in Maude, as we
can formalize the expression level as an equational theory and then using rewriting
rules to describe the concurrent features, i.e., the system level.

The system level comprises a rewriting rule for each concurrent feature and a
rewriting rule $\tau$ for sequential operations. While it would have been
possible to
define the sequential operations as an equational theory also at the system level, we take a different approach. Indeed, using rule $\tau$ is the only way to evaluate expressions (relying on the equational theory), but it forces evaluation to stop when some base cases are reached. This is more suitable to define the behaviour of a reversible debugger, which is our intended application. Notably, also a different semantics where expressions are fully evaluated could be made reversible using the approach we will describe in the next section.

Before presenting the rewriting logic, let us
discuss the entities that compose an Erlang system.
Processes are defined as tuples of the form:
\[\langle p, \theta, e, me \rangle\]
where $p$ is the process pid, $\theta$ is the environment binding variables to
values\footnote{Actually $\theta$ is a stack of environments, later on we will clarify
  why. }, $e$ is the expression currently under evaluation and $me$ is the
module environment, which contains the definitions of the functions declared in the module, that $p$ can invoke or
spawn.
Messages instead are defined as tuples of the form:
\[\langle p, p', v \rangle\]
where $p$ is the pid of the sender, $p'$ is the pid of the receiver and $v$ is
the payload. In the scope of this work processes and messages are entities in the
lower level of the semantics. We denoted them as $P$ in Section~\ref{sec:gener-appr-derive-rev-sem}.

A running system is composed of messages and processes, using the
parallel operator.

Now, let us analyze in detail the shape of the corresponding rewriting logic by
first analyzing the equational theory for expressions.

\subsection{Equational Theory}
The theory is defined as a set of equations which have one of the following generic forms:

\[
  \begin{array}{l}
    \ms{eq}:~[equation-name]\\
    ~~~~\langle l,\theta,~e \rangle = \langle l',\theta',~e' \rangle\\[2ex]

    \ms{ceq}:~[equation-name]\\
    ~~~~\langle l,\theta,~e \rangle = \langle l'',\theta'',~e'' \rangle\\
    ~~~~\ms{if}~\langle l', \theta', e'\rangle :=op(l,\theta,e)~\wedge~\langle
    l'',\theta'',~e'' \rangle := \langle l',\theta',e'\rangle

  \end{array}
\]

As we can see from above, to evaluate an expression we also need two additional items:
an environment $\theta$ and a label $l$. The environment binds each variable to its
value, if any. The label plays two roles:
i) communicates the kind of side effect performed by the expression, if any; ii) communicates information of the details of the side effect back and
forth between the expression level and the system level. Examples of this mechanism will be soon introduced.

Two kinds of equations are used: conditional ones, featuring an
$\ms{if}$ clause, and unconditional ones. Unconditional equations
define a simple reduction of the expression and change the label to
the appropriate one.

\begin{example}[Equation for self]
The unconditional equation below describes the behavior of \verb+self+ at the expression level.
\begin{verbatim}
eq [self] :
    < self(pid(INT)), ENVSTACK, atom("self")() > =
    < tau, ENVSTACK, int(INT) > .
\end{verbatim}
It reads roughly as follows: if the system level asks to check whether a \verb+self+ can be performed, communicating that the pid of the current process is \verb+INT+ (via \verb+self(pid(INT))+) and the expression is actually a self (\verb+atom("self")()+) then the expression reduces to the pid (\verb+int(INT)+) and the label becomes \verb+tau+, denoting successful evaluation.
\end{example}

Conditional equations can: either define a single step, that requires
some side condition (e.g., binding a variable to its value), or
perform some intermediate operation (e.g., selecting an inner
expression to evaluate) and then use recursively other equations (with the clause \verb+t:=t'+) to
reach a canonic form.

\begin{figure}[t]
  \centering
\begin{verbatim}
  ceq [receive] :
    < req-receive(PAYLOAD), ENV : ENVSTACK, receive CLSEQ end> =
    < received, ENV' : (ENV : ENVSTACK), begin EXSEQ end>
    if #entityMatchSuccess(EXSEQ | ENV') := 
       #entityMatch(CLSEQ ; #empty-clauselist | PAYLOAD | ENV ) .
\end{verbatim}
  \caption{Conditional equation for receive}
  \label{fig:eq-rec}
\end{figure}

\begin{example}[Conditional equation for receive]\label{ex:eqrec}
Figure~\ref{fig:eq-rec} describes the conditional equation for
receive. It reads roughly as follows. If the system level asks whether
a message with a given payload can be received
(\verb+req-receive(PAYLOAD)+) and the current expression is a receive
(\verb+receive CLSEQ end+) then one tries to match the body
\verb+CLSEQ+ of the receive against the payload using the environment \verb+ENV+. If the match succeeds then it returns the selected clause of the receive \verb+EXSEQ+ as well as the environment enriched with the bindings from the match
 \verb+ENV'+. In this case the expression can reduce to \verb+EXSEQ+ and will be evaluated in the new environment. Label \verb+received+ denotes successful reception.
\end{example}

\subsection{Rewriting Rules}
Let us now focus on rewriting rules, which have the following general shape:

\[
  \begin{array}{l}
    \ms{crl}:~[rule-name]\\
    ~~~~\langle p, \theta, e, me \rangle \paral E => \langle p,\theta',~e', me \rangle \paral op(l', E)\\
    ~~~~\ms{if}~\langle l', \theta', e'\rangle := \langle l,\theta,e\rangle
  \end{array}
\]

In the schema above $E$ captures other entities of the system, if any,
that may have an impact on the reduction, in particular a message that
may be received.  Rewriting rules are always conditional, as we always
rely on the expression semantics to understand which action the
selected process is ready to perform. Finally, we use $op$ to apply
side effects to $E$, determined by the label $l'$ produced by the
expression level. Examples \ref{ex:send} and~\ref{ex:rec} below show
some sample rewriting rules.

\begin{figure}[t]
  \centering
\begin{verbatim}
  crl [sys-send] :
      < P | exp: EXSEQ, env-stack: ENV, ASET > =>
      < P | exp: EXSEQ', env-stack: ENV', ASET > ||
      < sender: P, receiver: DEST, payload: GVALUE >
      if < DEST ! GVALUE, STORE', EXSEQ' > :=
         < req-gen, ENV, EXSEQ > .
\end{verbatim}
  \caption{System rule send}
  \label{fig:rule-send}
\end{figure}

\begin{example}\label{ex:send}
  Let us consider the conditional rewriting rule in Fig.~\ref{fig:rule-send}, which is used to send a message. In the conditional part of the
  rule we use the equational theory to check which kind of reduction the current expression \verb+EXSEQ+ can perform. If it can perform a send of a \verb+GVALUE+ to \verb+DEST+, then the process evolves so to evaluate the new expression \verb+EXSEQ'+ in the new environment \verb+ENV'+, and the new message is added to the system. Note that \verb+P+ is the pid of the process (Maude needs a unique identifier for this notation), and \verb+ASET+ includes other elements of the process which are not relevant here (currently, only the module environment).
  With respect to the general schema described above, here $E$ on the left-hand side is empty, and
  on the right-hand $op$ will add the message to $E$.
  This exemplifies how the label serves to communicate information from the
  expression level to the system one. Using this information, side effects (in this case the send of a message) are performed at the system level.
\end{example}

\begin{example}\label{ex:rec}
  Let us consider rule $\ms{sys-receive}$ in Fig.~\ref{fig:rule-rec}, which 
  is applied when a process receives a message. In the rule, if there is a
  message targeting the process, we use the equational theory (see
  Example~\ref{ex:eqrec}) to check whether the message can be received in the
  current state. If this is the case then the state is updated and the message
  is removed from the system. This rule shows how the label can be used to
  bubble up information from the system level the expression one.
  %% The preconditions
  %% for the successful firing of the rule are that the message targeting the
  %% selected process is floating in the soup and that the process is performing a
  %% receive with a clause aimed at receiving the message. What we do is bubbling
  %% up the payload of one of the (potential many) messages ready to be delivered
  %% to \verb_P_ with the label \verb_req-receive(GVALUE)_, then, if there is a
  %% receive statement to be evaluated and one of its clauses matches the message,
  %% we already have all the information necessary to select the appropriate branch
  %% locally, as can be seen in the equation $\ms{receive}$. Here, to fit the
  %% concrete instance of the rule in to the generic schema presented above we need
  %% to instantiate the $E$ on the left-hand side as the message to be received.
  %% Then, the operator expressing the side effect corresponds to the removal of
  %% the message from the system.
\end{example}

\begin{figure}[t]
  \centering
\begin{verbatim}
  crl [sys-receive] :
    < P | exp: EXSEQ, env-stack: ENV, ASET > ||
    < sender: SENDER, receiver: P, payload: GVALUE > =>
    < P | exp: EXSEQ', env-stack: ENV', ASET >
  if < received, ENV', EXSEQ' > :=
     < req-receive(GVALUE), ENV, EXSEQ > .
\end{verbatim}
  \caption{System rule receive}
  \label{fig:rule-rec}
\end{figure}

%Since no information from the
%system level are needed on the expression one to perform a send the starting
%label is a generic one, i.e., \verb_req-gen_.

\subsection{Design Choices}
%% Recursive expression level rules, conditional equations?

%% Avoiding production of illegal expressions
One of the difficulties of formalizing Erlang lies in the manipulation of the
expression, in fact a naive management could produce unwanted results or illegal
expression.

For example, consider
\begin{equation}\label{funcall}
  X=pow\_and\_sub(N,M) 
\end{equation}
where
\[
  pow\_and\_sub(N,M) \rightarrow Z = N*N, Z-M.
\]

Simply replacing the function call with its body would produce
\[
  X = Z = N*N, Z-M.
\]
which is a syntactical erlang expression but which would not have the desired effects, as the variable $X$ would assume the
value of $N*N$ and not the one of $Z-M$.

While dealing with constructs that produce a sequence of expressions
we also need to be careful as we may produce illegal terms. Consider the
following erlang expression

\begin{equation}\label{eq:case}
  \ms{case}~pow\_and\_sub(N,M)~of \ldots
\end{equation}

Simply replacing the function call with its body would produce 
\[
  \ms{case}~Z = N*N, Z-M~of \ldots
\]
which is illegal and would be refused by an erlang compiler.

The solution that we adopted, to solve both problems, consists of wrapping the produced sequence of
expressions with the construct \verb+begin_end+, which is the corresponding in
Erlang of parenthesis. For instance in~(\ref{funcall}) the produced expression
would be 
\[
  X = \ms{begin}~Z = N*N, Z-M~\ms{end}.
\]
and in this case $X$ is bound to the result of $Z-M$. This solution would
produce the desired effect even in a real Erlang environment.

About \ref{eq:case} the produced expression would be
\[
  \ms{case}~\ms{begin}Z = N*N, Z-M\ms{end}~of \ldots
\]
which is a correct erlang term, accepted by the compiler, in fact wrapping a sequence of expressions with erlang parenthesis
transforms the sequence in a single expression, from a syntactical point of
view.

Adding parenthesis requires us to properly manage the environment
corresponding to each block of code. So, rather than having a single
environment inside each process, we have a stack of environments, hence the $\theta$ in reality has the
shape $\theta_1:\ldots:\theta_n$. To sum up, each
time an expression $e$ that produces a sequence of expressions $e_1,\ldots,e_n$ is encountered we wrap
the sequence with $\ms{begin}~e_1,\ldots,e_n~\ms{end}$ and we push on the stack
of environments the appropriate environment to evaluate $e_1,\ldots,e_n$. 

In case $e$ is a function call the
appropriate environment is a new environment where eventual formal parameters of
the function are bound to the actual ones, while if $e$ is a $\ms{case}$ or a
$\ms{receive}$ statement then the appropriate environment is the previous one enriched
with the eventual variables that got bounded during the pattern matching.
Fig.~\ref{fig:rule-rec} shows the concrete \verb+receive+ meta-rule.

Finally, once the subcomputation is over and the expression has the shape $\ms{begin}~v~\ms{end}$ we simply replace it
with $v$ and we pop one environment from the stack. 

\section{Generating the Reversible Semantics}\label{sec:generating}

To produce the reversible semantics we decided to remain within Maude,
for two main reasons. First, Maude is well-suited to define program transformations
thanks to its META-LEVEL module which contains facilities to meta-represent a
module and to manipulate it. Second, since we defined Erlang's semantics inside
Maude we do not need to define a parser for it as it is already loaded and can
be easily meta-represented by taking advantage of Maude's facilities.

\subsection{Format Of The Non-Reversible Semantics}
As in~\cite{LaneseM20}, the input semantics must follow a given format, so that we
have a docking point.

Let us describe such format. First, it must have a module named \verb+SEM_SYSTEM+, which must contain the definition
of the top-level operators, where the operators defined correspond to the upper level
of the syntactical productions, as discussed
in~\ref{sec:gener-appr-derive-rev-sem} and the empty-system.
All the operators inside the module \verb+SEM_SYSTEM+ must be of sort
System' and their inputs must of sort 'System' as well. The subsort
relation 'Entity' \textless 'System' must be declared as well, in fact it is necessary
to use entities of the lower level inside the operators defined in the module.
This implies that the sort of the lower level operators must extend the sort
Entity.

\begin{figure}[t]
\begin{Verbatim}
mod SEM_SYSTEM is 
...
  sort Sys .
  subsort Entity < Sys .

  op #empty-system : -> Sys [ctor] .
  op _||_ : Sys Sys -> Sys [ctor assoc comm .. ] .
...
endm
\end{Verbatim}
  \label{fig:sem-system}
  \caption{Example of the system module for Erlang.}
\end{figure}

In Fig.~\ref{fig:sem-system} it is depicted the system module for the Erlang
language. We omitted elements that are not interesting in this context.

The rewriting rules of the rewriting theory that defines the
single steps of the reduction semantics must be defined under the module
\verb+SEM_TRANSITIONS+. Examples of the rules can be found in Fig.~\ref{fig:rule-send} and~\ref{fig:rule-rec}.

\subsection{Transformation}

The first transformation consists of adding a new \emph{sort} to the module
\verb+SEM_SYSTEM+,
i.e., the sort 'EntityWithKey'. The sort 'EntityWithKey' is built by composing
an entity of the lower level of the semantics and a key. Keys play the role
described in~\ref{sec:gener-appr-derive-rev-sem}. Then the subsort relation 'EntityWithKey'
\textless 'Entities' is added to the module, so that now all the top-level
operators can deal with tagged entities. 

Also, a new sort 'Placeholder' is declared and a new operator that when given a key
builds a placeholder is defined - this operator is what we called $C$ in~\ref{sec:methodology}.

Then, memories are added, by adding the sort 'Memory' and by defining the
operator that builds memories by combining a final configuration where the
entity have been replaced by their placeholders together with tagged entities,
which represent the state that gave rise to the step.

The transformation done here corresponds to the transformation done on the
syntactical productions in~\cite{LaneseM20}.

% A bit Technical, perhaps not needed?
The final transformation concerning the Entities module is to update the
equations to meta-represents and to lower the entities representation as now they 
are entities tagged with a key.

\subsection{Producing The Reversible Semantics}

The transformation to be performed over the rewriting rules again is the same as the
one described in~\ref{sec:methodology}. Rules now must deal with tagged entities and
each time that a forward step is taken the resulting entities must be tagged
with unique keys and the appropriate memory must be created.

The transformation is mostly forward, the only tricky part concerns the
uniqueness of the keys and their generation. Indeed, we must have a 'distributed' way to compute
them, as passing around a key generator would produce spurious dependencies. To
solve the problem we resorted to the following idea. Keys are represented as lists of
integers. Each time we need to produce new unique keys, to tag the (new) entities on
the right-hand side of a rule, we first collect all the keys on
the left-hand side of the rule, then we concatenate them together to a new list, say
$l$, and finally we tag
each of the new entities with $l$ concatenated with a different number for each
entity.

In practice, to transform a rule, we tag each entity of the left-hand
side with a variable that represents the key and we keep track of the variables
used. Then, to transform the right-hand side we tag all the created entities as
described above, using the keys of which we kept track, and we also introduce the memory created by the rule application.

In Fig.~\ref{fig:revsend} it is possible to observe rule \verb_send_ before and after being
transformed. As one can see, on the left-hand side, of the reversible version, the process is initially
tagged with some key, then the new entities on the right side are tagged with
fresh keys, built from the one already available locally. Moreover, now each
application of the rule will also produce a memory binding the two states.

\begin{figure}[t]
  \centering
\begin{verbatim}
  crl [sys-send] :
      < P | exp: EXSEQ, env-stack: ENV, ASET > =>
      < P | exp: EXSEQ', env-stack: ENV', ASET > ||
      < sender: P, receiver: DEST, payload: GVALUE >
      if < DEST ! GVALUE, ENV', EXSEQ' > :=
         < req-gen, ENV, EXSEQ > .

crl [label sys-send]:
    < P | ASET, exp: EXSEQ, env-stack: ENV > * key(N)
    => < sender: P, receiver: DEST, payload: GVALUE > * key(0 N) || 
       < P | exp: EXSEQ', env-stack: ENV', ASET > * key(1 N) || 
       [< P | ASET, exp: EXSEQ, env-stack: ENV > * key(N) ;
        @: key(0 N) || @: key(1 N)]
     if < DEST ! GVALUE, ENV', EXSEQ' > := < req-gen, ENV, EXSEQ > .
\end{verbatim}
  \caption{Forward And Forward Reversible Rule Send}
  \label{fig:revsend}
\end{figure}

The production of the backward semantics is even more straightforward of the
forward one, in fact to produce it it is sufficient swap the left-hand side of
each forward reversible rule with its right-hand side and get rid of the
conditional branch.

The conditional branch is not required anymore because if the process has
performed the forward step then it can always perform the backward one, given
that all the consequences of the action have been already undone. This last
condition is 'automatically' ensured by the presence of the memory together with
the entities bound by the placeholder inside the memory itself.

\begin{figure}[t]
  \centering
\begin{verbatim}
rl [label sys-send]:
   < sender: P, receiver: DEST,payload: GVALUE > * key(0 N) || 
   < P | exp: EXSEQ', env-stack: ENV', ASET > * key(1 N) || 
   [< P | ASET, exp: EXSEQ, env-stack: ENV > * key N ;
     @: key(0 N) || @: key(1 N)]
   => < P | ASET, exp: EXSEQ, env-stack: ENV > * key N 
\end{verbatim}
  
  \caption{Reversible backward rule send}
  \label{fig:revsend}
\end{figure}

\section{Ongoing Work}\label{sec:ongoing-work}

The last step to conclude this work is to close the gap between the format of
the rules expected by the general method and the actual format of the rules
provided in input. In fact, the schema of the general method allows for an
arbitrary number of rules (potentially infinite) describing the internal steps
of the system. Obviously, to efficiently describe a system we cannot produce
infinite rules, actually quite the contrary, i.e., the less we have the better
it is. So, in the formalization of the rules we resorted to the expression level
semantics. In particular, we used predicates - i.e., the expression level semantics - to gather
under the same meta-rule all infinite rules that would describe the same
behavior.

Let us consider the following processes:
\[
  \begin{array}{l}
    \langle p, \theta, 2~\ms{!}~'hello', me\rangle\\
    \langle p', \theta', \ms{case}~2~\ms{!}~'hello'~\ms{of} \ldots, me'\rangle\\
    \langle p'', \theta'', X = 2~\ms{!}~'hello', me''\rangle 
  \end{array}
\]

The three processes above are all ready to perform the same action, even though
they have a different configuration, nonetheless thanks to the expression level
semantics we are able to formalize their behavior in one meta-rule send.

Finally, what is required is a (set of) correctness theorem(s) to prove that all
the instances of the meta-rules used to formalize the languages are correct
rules in the sense required from~\cite{LaneseM20} so that all the backward rules
derived are also correctly modeling the backward semantics.

\section{Related}\label{sec:related}
First, in~\cite{Gonzalez-AbrilV21} the expression semantics is not directly
defined on expressions, but on an
operator $C[\bullet]$, which is used to select the part of the expression that has to be
evaluated, however no definition of such operator is given.

In Maude, since the semantics is
runnable, we need to define every operator. So, we opted for an approach closer
to the one in~\cite{LaneseNPV18}, where we have rules that recursively evaluate
the expression until a computation is executed - as already discussed, we can use
conditional equations to first perform operations and then use others equations
to further reduce.


For example, if the expression under evaluation is
$\ms{case}~foo(42)~\ms{of}\ldots$ the context operator would put automatically
the focus on $foo(42)$, i.e., $C[foo(42)]$. The convenience of using such
operator is that in this way one can avoid the definition of all the rules to
evaluate inner parts of an expression, like in the example just presented.

To avoid these problems the solution adopted
in~\cite{Gonzalez-AbrilV21} is to push the current environment together with the
context operator, where the term under evaluation is replaced with a
placeholder, in a stack and to start a new subcomputation - with the appropriate
environment. Then, once the
subcomputation is over the old contest and environment are retrieved from the
stack and the final value of the subcomputation is replaced inside the expression
which gave rise to it.

Again the context operators makes the formalization of this mechanism
much easier. Indeed in reality, to implement such approach it would be necessary to define, for each supported construct, a rule to start the subcomputation and another rule to replace the
value once the subcomputation is over.

To address the problem the solution proposed in~\cite{Gonzalez-AbrilV21},
already used in~\cite{LaneseNPV18}, is to evaluate those rules that need some information
available at the system level to a symbol $\kappa$, in other words a
\emph{future}. Then, the duty of replacing $\kappa$ with the
appropriate information is delegated to the system level.

The final difference between our semantics and the one proposed
in~\cite{Gonzalez-AbrilV21} concerns those rules of the expression semantics
that need to rely on some information held by the system level. This is the case
for instance for the $\ms{self}$ primitive or the $\ms{receive}$ statement, in
the first case the information required is the process pid, held in the process
tuple, and in the second case the information required is the message which has
to be received.



\section{Conclusion}\label{sec:conclusion}
To conclude the technical report let us now briefly resume the work done so far
and quickly discuss some future items.

First, we presented a new formalization of the Erlang language using Maude.
Having a mechanized version of the Erlang semantics makes it much easier to
debug it and to get convinced that the semantics correctly captures the
behavior of the language. Indeed, to test the semantics, in our work, one can
load an Erlang module and actually run an arbitrary complex program (as long as
the program uses supported primitives) and the various states traversed can be
compared with an actual execution to make sure that the first adheres to the
latter.

Concretely formalizing a languages poses also some challenges that
usually do not rise. For instance, it is often the case that 'theoretical
semantics' are not directly implementable, in fact to keep them compact and
elegant sometimes some operators are not fully defined or sometimes some
parameters appears out of the blue. In section~\ref{sec:formalizing-erlang} we
have discussed some examples.

Conversely, while defining a mechanized semantics it is not possible to leave
operators undefined or make parameters appearing out of the blue. This
constraint forced us to get creative during the formalization process, to
contain the number of rules, e.g., using labels to communicate information back
and forth between the two layers of the semantics.

Second, we implemented a program able to transform a non-reversible semantics in to a
reversible one, providing an implementation of the general method described
by~\cite{LaneseM20}. One interesting detail of the implementation, where again we had to be
creative, is the implementation of keys. In their work, due to the abstract
setting, to generate new keys the authors resorted to the use of a predicate
to generate fresh keys. In a concrete setting such predicate must be fully
defined, to do so we implemented keys as list of integers so that it would be
possible to generate new unique identifiers in a ``distributed manner''.

To conclude, let us discuss a couple of possible future directions for the presented
work. First, one could investigate ways to optimize the implementation of the
semantics, for example the way in which environments are managed is very naive,
the main advantage of doing so would be the ability of simulate even
computationally expensive erlang
programs. Second, one could expand the set of supported primitives to widen the
set of supported programs, in doing so of course it is important to make sure
that the causal-dependencies captured by the producer-consumer model are
appropriate - for example the model is not well-suited to capture read
dependencies.

\bibliographystyle{unsrt}
\bibliography{references.bib}

\end{document}
